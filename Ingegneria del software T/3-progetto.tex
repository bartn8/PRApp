\documentclass{article}
\usepackage[utf8]{inputenc}

\title{Ingegneria del Software T}
\author{
    Luca Bartolomei 
    \texttt{0000825005}
    \\
    Luigi Di Nuzzo
    \texttt{0000824873}
    \\
    Filippo Veronesi
    \texttt{0000832244}
}

\date{Marzo 2020}

\begin{document}

\maketitle

\tableofcontents

\newpage

\section{Abstract}

Il progetto riguarda la creazione di un applicativo software gestionale per prevendite elettroniche.\\
Abbiamo pensato il software per un gruppo di amici che organizzano feste, con obiettivi cardine l'abbattimento di costi, l'ottimizzazione dell'entrata dei partecipanti all'evento e una semplificazione dei conti di bilancio.\\
L'idea di fondo è di utilizzare come sostitutivo alla prevendita cartacea un codice QR in grado di far entrare il cliente dopo relativo check all'entrata.\\
Il risparmio economico ottenuto è ovviamente importante in confronto alla vendita tradizionale. Tuttavia bisogna tenere in conto dei problemi tecnologici che si possono verificare durante la vendita e l'entrata: problemi di connessione Internet, incompatibilità dei dispositivi dei clienti, training del personale addetto alle entrate, eccetera.\\
Viene gestito oltre alle prevendite e relatvi clienti, anche l'organizzazione e i vari membri dell'organizzazione, con divisione dei ruoli.\\
Per facilitare i conti di bilancio è disponibile una sezione in cui è possibile ricavare statistiche sull'andamento dell'evento.

\newpage

\section{Analisi dei requisiti}

\subsection{Requisiti del sistema}

\begin{itemize}
	
	%REQUISITI FUNZIONALI

    \item Il software prevede la possibilità di gestire più staff.
    \item Ogni staff gestito dal software è indipendente.
    \item La registrazione di utenti è libera.
    
	%-Requisiti di sicurezza
	
    %--Requisiti di identificazione
	\item Gli utenti devono essere identificati.
	
	%--Requisito di autenticazione
    \item Gli utenti sono autenticati tramite username e password.
    	
	%--Requisiti di immunità
    \item Si richiede una comunicazione sicura.
    \item Si richiedono metodi per cercare di garantire la disponibilità del servizio.
    
	%--Requisiti di Autorizzazione
    \item Ogni membro di uno staff può ricoprire dei ruoli: cassiere, PR, amministratore.
    \item Ogni ruolo è indipendente.
    
    \item Il ruolo di cassiere riguarda la timbratura di prevendite all'ingresso di un evento.
    \item Il ruolo di PR riguarda la vendita di prevendite a clienti.
    \item Il ruolo di amministratore riguarda la gestione dello staff, degli eventi e delle tipologie di prevendita di un evento.
	
	%--Requisiti di scoperta alle intrusioni
	\item Utilizzo di log per monitorare lo stato.
	
	
	%--Requisiti di non-ripudiabilità
	\item Quando un PR vende una prevendita viene associata ad esso.
	
	%--Requisiti di riservatezza
	\item Le password devono essere salvate (se necessario) in modo sicuro.
	\item Il log deve essere salvato in modo sicuro.
	
	%--------------------
    
    \item Ogni utente registrato nel gestionale può diventare membro di uno o più staff, solo se autorizzato. 
    \item L'accesso ad uno staff, da parte di un utente, avviene tramite codice di accesso.
    \item Ogni utente registrato può creare al massimo uno staff.
    \item Quando un utente crea uno staff ne diventa amministratore.
    
    
    
    
    \item La timbratura di una prevendita valida permette al cliente di entrare all'evento, ovviamente lo staff potrà effettuare ulteriori controlli non previsti dal sistema e decidere di far entrare un cliente.
    \item La timbratura può essere fatta solo una volta.
    
    \item La vendita di una prevendita elettronica consiste nella consegna al cliente di un documento digitale di qualche forma, associato alla prevendita elettronica venduta. Il cliente sceglierà la tipologia di prevendita.
    \item Il documento digitale consegnato al cliente sarà utilizzato dal cassiere all'ingresso timbrare la prevedita elettronica.
    \item Si prevedono più forme equivalenti di documenti digitali, per affrontare le eterogeneità.
    
    
    \item La gestione di staff riguarda le autorizzazioni concesse ai membri, rimozione di membri, statistiche di un membro, statistiche dell'evento, statistiche dello staff.
    
    \item Le statistiche di un membro sono suddivise per ruolo coperto all'interno dello staff: cassiere o PR.
    \item Il ruolo di amministratore non prevede statistiche personali.
    \item Gli utenti non amministratori possono vedere solo le statistiche personali.
    
    \item Per gestione degli eventi di uno staff si intende la possibilità di vedere gli staff di un evento, di creane uno nuovo e di poter modificare un evento dello staff.
    \item Gli utenti non amministratori possono solo vedere gli eventi dello staff.
    \item La gestione delle tipologie di prevedita di un evento indica l'aggiunta, la modifica e la rimozione delle tipologie di prevendita.
    \item Gli utenti non amministratori possono solo vedere le tipologie di prevendite associate ad un evento.
    
    
    \item Una tipologia di prevendita non è altro che un modello di prevendita con un prezzo, una descrizione e un periodo temporale anticipante l'evento in cui è possibile vendere la prevendita.
    
    \item Un evento è composto da un nome, una descrizione, un periodo temporale di svolgimento e un luogo. Inoltre ha anche uno stato che ne indica se l'evento è valido o annullato.
    
    \item Ogni prevendita venduta `e associata ad una sola tipologia di prevendita.
    \item Ogni prevedita `e nominativa.
    \item Una prevendita ha anche uno stato che ne indica se è stata consegnata, pagata, annullata o rimborsata.
    
    %REQUISITI NON FUNZIONALI
    
\end{itemize}

\newpage

\subsection{Analisi del dominio}

\subsubsection{Vocabolario}

\begin{center}
    \begin{tabular}{|c|c|c|}
        \hline
        \textbf{Voce} & \textbf{Definizione} & \textbf{Sinonimi} \\
        \hline\hline
        test & test & test \\
        \hline
    \end{tabular}
\end{center}

%Lo staff è un gruppo di utenti con lo scopo di creare eventi, vendere prevendite e registrare le entrate dell'evento.
%Un utente è una persona regitrata nel software gestionale, in grado di iscriversi ad uno staff, diventandone membro.

\end{document}
