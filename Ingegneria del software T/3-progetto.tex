\documentclass[a4paper]{article}
\usepackage[utf8]{inputenc}
\usepackage[margin=50pt]{geometry}
\usepackage{tabulary}

\title{Ingegneria del Software T}
\author{
    Luca Bartolomei 
    \texttt{0000825005}
    \\
    Luigi Di Nuzzo
    \texttt{0000824873}
    \\
    Filippo Veronesi
    \texttt{0000832244}
}

\date{Marzo 2020}

\begin{document}

\maketitle

\tableofcontents

\newpage

\section{Abstract}

Il progetto riguarda la creazione di un applicativo software gestionale per prevendite elettroniche.\\
Abbiamo pensato il software per un gruppo di amici che organizzano feste, con obiettivi cardine l'abbattimento di costi, l'ottimizzazione dell'entrata dei partecipanti all'evento e una semplificazione dei conti di bilancio.\\
L'idea di fondo è di utilizzare, come sostitutivo alla prevendita cartacea, un codice QR in grado di far entrare il cliente dopo relativo check all'entrata.\\
Il risparmio economico ottenuto è ovviamente importante in confronto alla vendita tradizionale. Tuttavia bisogna tenere in conto dei problemi tecnologici che si possono verificare durante la vendita e l'entrata: problemi di connessione Internet, incompatibilità dei dispositivi dei clienti, training del personale addetto alle entrate, eccetera.\\
Viene gestito oltre alle prevendite e relatvi clienti, anche l'organizzazione e i vari membri dell'organizzazione, con divisione dei ruoli.\\
Per facilitare i conti di bilancio è disponibile una sezione in cui è possibile ricavare statistiche sull'andamento dell'evento.

\section{Analisi dei requisiti}

\subsection{Requisiti del sistema}

\begin{itemize}
	
	%\item \textbf{REQUISITI FUNZIONALI}
	

	%Non funzionale
	\item Requisito fondamentale è il basso costo del prodotto software.
	
	\item Il software prevede la possibilità di gestire più staff.
	
	%-Requisiti di sicurezza
	
	%--Requisiti di identificazione
	\item Gli utenti devono essere identificati tramite username.
	
	%--Requisito di autenticazione
	\item Gli utenti sono autenticati tramite credenziali di username e password.
	
	%Non funzionale??
	\item Utilizzare uno o più metodi per velocizzare l'autenticazione dell'utente.
	
	\item L'accesso ad uno staff, da parte di un utente, avviene tramite codice di accesso.	
		
	%--Requisiti di immunità
	%Utilizzo https per la cifratura
	\item Si richiede una comunicazione sicura.
	
	%Sistema locale senza server remoto tramite server locale + wifi
	\item Si richiedono procedure manuali o automatiche per cercare di garantire la disponibilità del servizio.
	
	%Non funzionale??
	%Devo contare i tentativi falliti nella SESSIONE.
	\item Bloccare l'account utente dopo troppi tentativi di accesso e notificarlo nei log.
	
	\item Possibilità di cambiare il codice di accesso dello staff.
	\item La registrazione di utenti è a carico dell'amministratore di sistema.
	\item Possibilità di cambiare la password personale dell'utente.
	
	%Codice della prevendita
	%Prevendita nominativa
	\item Si richiedono metodi per evitare la contraffazione delle prevendite.
	
	%Non funzionale??
	\item La password fornita dall'amministratore di sistema a tempo di registrazione va cambiata immediatamente dopo il login.
	
	%--Requisiti di Autorizzazione
	%Non funzionale??
	\item Ogni staff gestito dal software è indipendente.
	
	\item Ogni membro di uno staff può ricoprire dei ruoli: cassiere, PR, amministratore.
	
	%Non funzionale??
	\item Ogni ruolo è indipendente.
	
	\item Il ruolo di cassiere riguarda la timbratura di prevendite all'ingresso di un evento.
	\item Il ruolo di PR riguarda la vendita di prevendite a clienti.
	\item Il ruolo di amministratore riguarda la gestione dello staff, degli eventi e delle tipologie di prevendita di un evento.
	
	%--Requisiti di scoperta alle intrusioni
	\item Utilizzo di log per monitorare operazioni critiche.
	\item Prevedere livelli di log per aiutare l'analisi da parte dell'amministratore.
	
	%--Requisiti di non-ripudiabilità
	\item Quando un PR vende una prevendita, essa viene associata ad esso.
	\item Quando un Cassiere timbra una prevendita, essa viene associata ad esso.
	
	%--Requisiti di riservatezza
	\item Le password devono essere salvate in modo sicuro.
	
	%Non vuol dire cifrato! Basta un controllo agli accessi.
	%Se salvi in DB ricorda di indicizzare.
	%\item Il log deve essere salvato in modo abbastanza sicuro.
	
	%--------------------
	
	%Accesso e creazione di uno staff
	\item Ogni utente registrato nel gestionale può diventare membro di uno o più staff. 
	\item Ogni utente registrato può creare al massimo uno staff.
	\item Quando un utente crea uno staff ne diventa membro e amministratore.
	
	%Timbratura (Cassiere)
	\item La timbratura di una prevendita valida permette al cliente di entrare all'evento, ovviamente lo staff potrà effettuare ulteriori controlli non previsti dal sistema e decidere di far entrare un cliente.
	\item La timbratura può essere fatta solo una volta.
	
	%Vendita (PR)
	\item Prima della vendita il cliente sceglierà una tipologia di prevendita associata all'evento a cui vuole partecipare.
	
	%Non viene specificata la modalità di consegna: URL, file immagine file testo etc.
	\item La vendita di una prevendita elettronica consiste nella consegna al cliente di un documento digitale di qualche forma, associato alla prevendita elettronica venduta.
	\item Il documento digitale consegnato al cliente sarà utilizzato dal cassiere all'ingresso timbrare la prevedita elettronica.
	
	%Non funzionale???
	\item Si prevedono più forme di consegna del documento digitale, per affrontare le eterogeneità.
	
	%Gestione staff (Amministratore)
	\item La gestione di staff riguarda la gestione dei ruoli concessi ai membri, rimozione di membri, statistiche di un membro, statistiche dell'evento, statistiche dello staff.
	
	\item Un amministratore può concedere/revocare i ruoli a qualsiasi membro dello staff. Unico vincolo è che rimanga almeno un amministratore.
	   
	\item Per gestione degli eventi di uno staff si intende la possibilità di vedere gli staff di un evento, di creane uno nuovo e di poter modificare un evento dello staff.
	
	%Non funzionale???
	\item Gli utenti non amministratori possono solo vedere gli eventi dello staff.
	
	\item La gestione delle tipologie di prevedita di un evento indica l'aggiunta, la modifica e la rimozione delle tipologie di prevendita.
	
	%Non funzionale???
	\item Gli utenti non amministratori possono solo vedere le tipologie di prevendite associate ad un evento.
	
	%Definizioni
	%-Eventi
	\item Un evento è composto da un nome, una descrizione, un periodo temporale di svolgimento e un luogo. 
	
	%Funzionale/Non funzionale???
	\item Un evento può essere annullato, anche se ci sono prevendite vendute. Un evento annullato rimane tale.
	
	%Tipologia Prevendita
	\item Una tipologia di prevendita serve ad associare alla prevendita un prezzo e una descrizione a tutte le prevendite con la stessa tipologia.
	
	%Funzionale/Non funzionale???
	\item Una tipologia di prevendita contiene un periodo temporale precedente al periodo di svolgimento dell'evento, dentro al quale è possibile vendere le prevendite associate alla tipologia.
	
	%-Prevendite
	\item Ogni prevendita venduta è associata ad una sola tipologia di prevendita.
	\item Ogni prevedita è nominativa.
	
	%Funzionale/Non funzionale???
	\item Una prevendita può essere annullata e/o rimborsata. Una prevendita annullata e/o rimborsata rimane tale.
	
	%-Statistiche
	\item Le statistiche di un membro sono suddivise per ruolo coperto all'interno dello staff: cassiere o PR.
	\item Il ruolo di amministratore non prevede statistiche personali.
	
	%Non funzionale???
	\item Gli utenti non amministratori possono vedere solo le statistiche personali.	
	
    
    %\item \textbf{REQUISITI NON FUNZIONALI}
	\item Il blocco dell'account deve avvenire dopo 3 tentativi.
	\item La password degli utenti deve essere lunga almeno 8 caratteri.
	\item Il codice di accesso allo staff deve essere lungo almeno 4 caratteri.
	
	
\end{itemize}

\newpage

\subsection{Analisi del dominio}

\subsubsection{Glossario}

\begin{table}[ht!]
  \begin{center}
    \begin{tabulary}{1\textwidth}{c|C|C}
        \textbf{Voce} & \textbf{Definizione} & \textbf{Sinonimi}\\
        \hline
        \hline
		Amministratore di sistema & Utente con privilegi di sistema aggiuntivi. & \\
		\hline
		Privilegio di sistema & Autorizzazione intrinseca concessa ad un amministratore di sistema che riguarda la gestione del software stesso. Non riguarda gli staff. & \\
		\hline
        Staff & Gruppo di utenti con lo scopo di organizzare eventi. & Ente organizzatore \\
        \hline
        Utente & Persona registrata nel software gestione. & \\
        \hline
		Cliente & Persona che vuole partecipare ad un evento di uno staff & \\
		\hline
        Membro & Utente che è iscritto ad uno staff. & Organizzatore \\
        \hline
		PR & Membro di uno staff che si occupa della vendita di prevendite & \\
		\hline
		Cassiere & Membro di uno staff che si occupa dell'entrata dei clienti ad un evento & \\
		\hline
		Amministratore & Membro di uno staff che si occupa della gestione dello staff steso & \\
		\hline
        Ruolo & Autorizzazione che ha il membro all'interno dello staff & Autorizzazione \\
        \hline
        Evento & Avvenimento registrato dallo staff, per il quale è possibile vendere prevendite e registrare ingressi & Festa \\
        \hline
        Tipologia Prevendita & Modello associato ad un evento, la quale da le caratteristiche di prezzo e descrizione alla prevendita venduta & Tipo Prevendita \\
        \hline
        Prevendita & Biglietto venduto anticipatamente, che consente l'entrata all'evento pagato & Ticket, Prevendita Elettronica \\
        \hline
		Statistiche & Informazioni di carattere gestionale, riguardo ad un evento o a un membro dello staff & \\
		\hline
		Documento digitale & Si tratta di una risorsa digitale, reperibile dal cliente, che serve a identificare una prevendita venduta & \\
		\hline
		Log & Registro dove vengono salvate informazioni per risalire ad operazioni critiche svolte & \\
		\hline
		Operazione & Comando richiesto al software gestionale da parte di un utente & \\
		\hline
		Login & Operazione per identificare un utente & Accesso utente \\
		\hline
		Timbratura & Operazione svolta da un cassiere svolta per validare una prevendita di un cliente & Convalida della prevendita \\
		\hline
		Credenziali & Coppia di valori username e password utilizzata per l'autenticazione dell'utente & \\
		\hline
		Username & Stringa di caratteri alfanumerici. Serve a identificare l'utente & \\
		\hline
		Password & String di caratteri alfanumerici. Può contenere caratteri speciali. & \\
		\hline
    \end{tabulary}
  \end{center}
\end{table}

\newpage

\subsection{Casi d'uso}


\subsection{Scenari}





\end{document}
