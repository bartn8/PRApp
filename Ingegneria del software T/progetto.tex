\documentclass{article}
\usepackage[utf8]{inputenc}

\title{Ingegneria del Software T}
\author{
    Luca Bartolomei 
    \texttt{0000825005}
    \\
    Luigi Di Nuzzo
    \texttt{0000824873}
    \\
    Filippo Veronesi
    \texttt{0000832244}
}

\date{Marzo 2020}

\begin{document}

\maketitle

\tableofcontents

\section{Abstract}

Il progetto riguarda la creazione di un applicativo software gestionale per prevendite elettroniche.\\
Abbiamo pensato il software per un gruppo di amici che organizzano feste, con obiettivi cardine l'abbattimento di costi, l'ottimizzazione dell'entrata dei partecipanti all'evento e una semplificazione dei conti di bilancio.\\
L'idea di fondo è di utilizzare come sostitutivo alla prevendita cartacea un codice QR in grado di far entrare il cliente dopo relativo check all'entrata.\\
Il risparmio economico ottenuto è ovviamente importante in confronto alla vendita tradizionale. Tuttavia bisogna tenere in conto dei problemi tecnologici che si possono verificare durante la vendita e l'entrata: problemi di connessione Internet, incompatibilità dei dispositivi dei clienti, training del personale addetto alle entrate, eccetera.\\
Viene gestito oltre alle prevendite e relatvi clienti, anche l'organizzazione e i vari membri dell'organizzazione, con divisione dei ruoli.\\
Per facilitare i conti di bilancio è disponibile una sezione in cui è possibile ricavare statistiche sull'andamento dell'evento.


\end{document}
